\documentclass[10pt,twoside,slovak,a4paper]{article}

\usepackage[slovak]{babel}
%\usepackage[T1]{fontenc}
\usepackage[IL2]{fontenc} % lepšia sadzba písmena Ľ než v T1
\usepackage[utf8]{inputenc}
\usepackage{graphicx}
\usepackage{url} % príkaz \url na formátovanie URL
\usepackage{hyperref} % odkazy v texte budú aktívne (pri niektorých triedach dokumentov spôsobuje posun textu)

\usepackage{cite}
%\usepackage{times}

\pagestyle{headings}

\title{Virtuálny lekár\thanks{Semestrálny projekt v predmete Metódy inžinierskej práce, ak. rok 2021/22, vedenie: Zuzana Špitálová}} % meno a priezvisko vyučujúceho na cvičeniach

\author{Jaroslav Samuel Griščík\\[2pt]
	{\small Slovenská technická univerzita v Bratislave}\\
	{\small Fakulta informatiky a informačných technológií}\\
	{\small \texttt{xgriscik@stuba.sk}}
	}

\date{\small 19. októbra 2020} % upravte



\begin{document}

\maketitle

\begin{abstract}

V posledných zopár desaťročiach monitorovanie zdravia a zdravotných informácií stvorilo mnoho vedeckých článkov a veľký záujem trhu. Toto je indikované vývojom rôznych nositeľných gadgetov a zdraviu  monitorovacích systémov, ako napríklad inteligentné hodinky, alebo merač hladiny inzulínu v krvi; ako aj výskumom a projektom zaoberajúcim sa bezpečnosťou výmeny zdravotných informácií. Tento článok sa bude zaoberať prototypom virtuálneho lekára pre diagnózu na báze dát získaných z nosenia už spomenutých gadgetov. Tieto dáta budú použité trénovanou a personalizovanou neurónovou sieťou pre diagnózu pacientovho zdravotného stavu. Ďalej sa bude zaoberať bezpečným prenosom pacientových dát a ich sprístupneniu reálnemu lekárovi ktorý ich dokáže oprávniť. 
\cite{2014}

\end{abstract}

\includegraphics[scale=0.5]{temakapitoly.png}

\bibliography{literatura.bib}
\bibliographystyle{plain}
\end{document}
